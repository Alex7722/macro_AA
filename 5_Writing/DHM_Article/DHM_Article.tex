% AER-Article.tex for AEA last revised 22 June 2011
\documentclass[JEL]{AEA}

% The mathtime package uses a Times font instead of Computer Modern.
% Uncomment the line below if you wish to use the mathtime package:
%\usepackage[cmbold]{mathtime}
% Note that miktex, by default, configures the mathtime package to use commercial fonts
% which you may not have. If you would like to use mathtime but you are seeing error
% messages about missing fonts (mtex.pfb, mtsy.pfb, or rmtmi.pfb) then please see
% the technical support document at http://www.aeaweb.org/templates/technical_support.pdf
% for instructions on fixing this problem.

% Note: you may use either harvard or natbib (but not both) to provide a wider
% variety of citation commands than latex supports natively. See below.

% Uncomment the next line to use the natbib package with bibtex
\usepackage{natbib}

% Uncomment the next line to use the harvard package with bibtex
%\usepackage[abbr]{harvard}

% This command determines the leading (vertical space between lines) in draft mode
% with 1.5 corresponding to "double" spacing.
\draftSpacing{1.5}


% tightlist command for lists without linebreak
\providecommand{\tightlist}{%
  \setlength{\itemsep}{0pt}\setlength{\parskip}{0pt}}


% Pandoc citation processing
\newlength{\cslhangindent}
\setlength{\cslhangindent}{1.5em}
\newlength{\csllabelwidth}
\setlength{\csllabelwidth}{3em}
\newlength{\cslentryspacingunit} % times entry-spacing
\setlength{\cslentryspacingunit}{\parskip}
% for Pandoc 2.8 to 2.10.1
\newenvironment{cslreferences}%
  {}%
  {\par}
% For Pandoc 2.11+
\newenvironment{CSLReferences}[2] % #1 hanging-ident, #2 entry spacing
 {% don't indent paragraphs
  \setlength{\parindent}{0pt}
  % turn on hanging indent if param 1 is 1
  \ifodd #1
  \let\oldpar\par
  \def\par{\hangindent=\cslhangindent\oldpar}
  \fi
  % set entry spacing
  \setlength{\parskip}{#2\cslentryspacingunit}
 }%
 {}
\usepackage{calc}
\newcommand{\CSLBlock}[1]{#1\hfill\break}
\newcommand{\CSLLeftMargin}[1]{\parbox[t]{\csllabelwidth}{#1}}
\newcommand{\CSLRightInline}[1]{\parbox[t]{\linewidth - \csllabelwidth}{#1}\break}
\newcommand{\CSLIndent}[1]{\hspace{\cslhangindent}#1}


\usepackage{hyperref}

\begin{document}


\title{DHM: a Digital History of Macroeconomics interactive platform}
\shortTitle{Digital History of Macroeconomics}
% \author{Author1 and Author2\thanks{Surname1: affiliation1, address1, email1.
% Surname2: affiliation2, address2, email2. Acknowledgements}}


\author{
  Aurélien Goutsmedt\\
  Alexandre Truc\thanks{
  Goutsmedt: ULC, \href{mailto:Goutsmedt@example.com}{Goutsmedt@example.com}.
  Truc: UCA (Université Côte d'Azur), CNRS (Centre national de la
recherche scientifique), GREDEG (Groupe de Recherche en Droit, Economie,
Gestion), 250 Rue Albert Einstein, 06560 Valbonne.
France., \href{mailto:Alexandre.Truc@unice.fr}{Alexandre.Truc@unice.fr}.
  We thank Till Duppe for his wisdom
}
}

\date{\today}
\pubMonth{06}
\pubYear{2022}
\pubVolume{1}
\pubIssue{1}
\JEL{A10, A11}
\Keywords{first keyword, second keyword}

\begin{abstract}
Abstract goes here
\end{abstract}


\maketitle

The Mapping Macroeconomics project is an online interactive platform
displaying bibliometric data on a large set of macroeconomic articles.
It aims at offering a better understanding of the history of
macroeconomics through the navigation between the different bibliometric
networks.

The point of departure of the project is the observation of an
exponential increase in the number of articles published in academic
journals in economics since the 1970s. This phenomenon makes it harder
for historians of economics to properly assess the trends in the
transformation of economics, the main topics researched, the most
influential authors and ideas, etc. We consider that developing
collective quantitative tools could help historians to confront this
challenge. The opportunities that a quantitative history brings are
particularly useful to the recent history of macroeconomics. Practicing
macroeconomists are eager to tell narratives of the evolution of their
field that serve the purpose of intervening on current debates, by
giving credit to particular authors and weight to specific ideas.
Historians who go into this area find plenty of accounts by
macroeconomists and have to handle the vast increase in the
macroeconomic literature since the last quarter of the past century. The
Mapping Macroeconomics platform aims at helping historians to
empirically check macroeconomists' narratives on the discipline, to
explore interesting patterns on the evolution of macroeconomics, and
eventually to write new histories of macroeconomics.

\section{Methodology}

\subsection{Construction of the Corpus}

Our corpus is composed of macroeconomic articles published in economics.
We identified all articles published in macroeconomics using JEL codes
related to macroeconomics (Econlit database). JEL codes are used in
economics to classify articles into specialties, like
``Microeconomics'', ``Macroeconomics \& Monetary Economics'',
``Industrial Organization'', etc. An article can have multiple JEL codes
and so can be identified as part of multiple specialties. The JEL
nomenclature was radically altered in 1991, and while these results in
some discontinuity between the two nomenclatures, there are some
correspondence (see Cherrier (2017) for a history of the JEL codes). The
contemporary list of JEL codes can be found on the AEA
website\footnote{\url{https://www.aeaweb.org/econlit/jelCodes.php}} and
the old JEL codes with old/new correspondence table can be found in the
Journal of Economic Literature, volume 29(1) (JEL, 1991).

For our corpus, we consider that an article is a macroeconomics article
if it has one of the following codes:

\begin{itemize}
\item
  For old JEL codes (pre-1991): 023, 131, 132, 133, 134, 223, 311, 313,
  321, 431, 813, 824.
\item
  For new JEL codes (1991 onward): all E, F3 and F4.2.
\end{itemize}

Two additional comments on the JEL classification are necessary:

\begin{itemize}
\item
  First, we had to use some pre-1991 JEL codes that are not considered
  in the new classification as totally belonging to macroeconomics.
  Consequently, many articles in our pre-1991 corpus are public
  finance/public economics articles. Nonetheless, this group is clearly
  identifiable in our networks and thus do not disturb the
  interpretation of our results.
\item
  Second, in the recent classification, the letter E designates
  macroeconomics JEL code, while F designates International Economics.
  In this last sub-discipline, we decided that it would be important to
  have articles dealing with international macroeconomics and thus we
  integrated articles with F3 and F4.2 JEL codes.
\end{itemize}

Using these JEL codes, we match the articles extracted from Econlit with
Web of Science articles using the following set of matching variables:

\begin{itemize}
\item
  Journal, Volume, First Page
\item
  Year, Journal, First Page, Last Page
\item
  First Author, Year, Volume, First Page
\item
  First Author, Title, Year
\item
  Title, Year, First Page
\end{itemize}

We expect that these matching procedure results in some false positive.
However, two elements prevent false positive from having any importance
on the platform. First, we applied a general threshold on edges by
keeping links between articles that had at least two references in
common, and our projected networks are only made of the main component
of our corpus (i.e,the biggest connected network). In other words, false
positive have a very high chance of being completely disconnected from
our main component and therefore filtered out from our analysis. Second,
even if false positive ``articles'' are present in some networks, these
articles, when irrelevant, would be relegated at the margins of our
networks and thus do not have any significant impact on our results.

\subsection{Network construction}

Our networks are based on bibliographic coupling. In a bibliographic
coupling network, a link is created between two articles when they have
one or more references in common. The more references two articles have
in common, the stronger the link. The idea is that articles sharing many
references to gather are likely to share cognitive content (ideas,
theories, methods, objects of study, etc.).

To normalize and weight the link between two articles, we used the
refined bibliographic coupling strength of Shen et al. (2019). This
method normalized and weight the strength between articles by
considering two important elements:

The size of the bibliography of the two linked articles. It means that
common references between two articles with long bibliography are
weighted as less significant since the likeliness of potential common
references is higher. Conversely, common references between two articles
with a short bibliography is weighted as more significant.

The number of occurrences of each reference in the overall corpus. When
a reference is shared between two articles, it is weighted as less
significant if it is a very common reference across the entire corpus
and very significant if it is scarcely cited. The assumption is that a
very rare common reference points to a higher content similarity between
two articles than a highly cited reference.

For all macroeconomics articles published in the EER and in the Top 5,
we build successive networks on 5-year overlapping windows (1969-1973;
1970-1974; \ldots; 2010-2014; 2011-2015). This results in 43 networks.

For each network:

We apply a general threshold on edges by keeping links between articles
that had at least two references in common before weighting. We consider
that it is more likely that the link between two articles is significant
if they share at least two references.

We only kept the main component of the network (thus ignoring singleton
and secondary components). Nonetheless, we take care to check that any
secondary component did not represented more than 2\% of the whole
network.

We place nodes in a 2-dimensional space using the Force Atlas 2
algorithm Jacomy et al. (2014)

(Jacrelied on an attracive force---bringing closer articles which are
linked---and a repulsive force---moving away the articles with no link,
while minimizing the crossing between edges. (Vedeld 1994)

The size of the nodes depends of the number of citations---coming from
other macroeconomics papers---the article received during the time
window.

We identify relevant groups of articles using a cluster detection
algorithm and colored nodes according to the cluster they belong to (see
below for details).

\section{Features}

Sample figure:

\begin{figure}
Figure here.

\caption{Caption for figure below.}
\begin{figurenotes}
Figure notes without optional leadin.
\end{figurenotes}
\begin{figurenotes}[Source]
Figure notes with optional leadin (Source, in this case).
\end{figurenotes}
\end{figure}

Sample table:

\begin{table}
\caption{Caption for table above.}

\begin{tabular}{lll}
& Heading 1 & Heading 2 \\
Row 1 & 1 & 2 \\
Row 2 & 3 & 4%
\end{tabular}
\begin{tablenotes}
Table notes environment without optional leadin.
\end{tablenotes}
\begin{tablenotes}[Source]
Table notes environment with optional leadin (Source, in this case).
\end{tablenotes}
\end{table}

\% The appendix command is issued once, prior to all appendices, if any.
\appendix

\section{Mathematical Appendix}

\hypertarget{refs}{}
\begin{CSLReferences}{1}{0}
\leavevmode\vadjust pre{\hypertarget{ref-cherrierClassifyingEconomicsHistory2017}{}}%
Cherrier, Beatrice. 2017. {``Classifying {Economics}: {A History} of the
{JEL Codes}.''} \emph{Journal of Economic Literature} 55 (2): 545--79.
\url{https://econpapers.repec.org/article/aeajeclit/v_3a55_3ay_3a2017_3ai_3a2_3ap_3a545-79.htm}.

\leavevmode\vadjust pre{\hypertarget{ref-jacomy_ForceAtlas2ContinuousGraph_2014}{}}%
Jacomy, Mathieu, Tommaso Venturini, Sebastien Heymann, and Mathieu
Bastian. 2014. {``{ForceAtlas2}, a {Continuous Graph Layout Algorithm}
for {Handy Network Visualization Designed} for the {Gephi Software}.''}
Edited by Mark R. Muldoon. \emph{PLoS ONE} 9 (6): e98679.
\url{https://doi.org/10.1371/journal.pone.0098679}.

\leavevmode\vadjust pre{\hypertarget{ref-shenRefinedMethodComputing2019}{}}%
Shen, Si, Danhao Zhu, Ronald Rousseau, Xinning Su, and Dongbo Wang.
2019. {``A Refined Method for Computing Bibliographic Coupling
Strengths.''} \emph{Journal of Informetrics} 13 (2): 605--15.
\url{https://doi.org/10.1016/j.joi.2019.01.012}.

\leavevmode\vadjust pre{\hypertarget{ref-vedeld_environment_1994}{}}%
Vedeld, Paul O. 1994. {``The Environment and Interdisciplinarity
Ecological and Neoclassical Economical Approaches to the Use of Natural
Resources.''} \emph{Ecological Economics} 10 (1): 1--13.
\url{https://doi.org/10.1016/0921-8009(94)90031-0}.

\end{CSLReferences}


\end{document}
